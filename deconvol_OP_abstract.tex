%% Copernicus Publications Manuscript Preparation Template for LaTeX Submissions
%% ---------------------------------
%% This template should be used for copernicus.cls
%% The class file and some style files are bundled in the Copernicus Latex Package, which can be downloaded from the different journal webpages.
%% For further assistance please contact Copernicus Publications at: production@copernicus.org
%% https://publications.copernicus.org/for_authors/manuscript_preparation.html


%% Please use the following documentclass and journal abbreviations for discussion papers and final revised papers.


%% 2-column papers and discussion papers
\documentclass[acp, manuscript]{copernicus}



%% Journal abbreviations (please use the same for discussion papers and final revised papers)

% Archives Animal Breeding (aab)
% Atmospheric Chemistry and Physics (acp)
% Advances in Geosciences (adgeo)
% Advances in Statistical Climatology, Meteorology and Oceanography (ascmo)
% Annales Geophysicae (angeo)
% ASTRA Proceedings (ap)
% Atmospheric Measurement Techniques (amt)
% Advances in Radio Science (ars)
% Advances in Science and Research (asr)
% Biogeosciences (bg)
% Climate of the Past (cp)
% Drinking Water Engineering and Science (dwes)
% Earth System Dynamics (esd)
% Earth Surface Dynamics (esurf)
% Earth System Science Data (essd)
% Fossil Record (fr)
% Geographica Helvetica (gh)
% Geoscientific Instrumentation, Methods and Data Systems (gi)
% Geoscientific Model Development (gmd)
% Hydrology and Earth System Sciences (hess)
% History of Geo- and Space Sciences (hgss)
% Journal of Micropalaeontology (jm)
% Journal of Sensors and Sensor Systems (jsss)
% Mechanical Sciences (ms)
% Natural Hazards and Earth System Sciences (nhess)
% Nonlinear Processes in Geophysics (npg)
% Ocean Science (os)
% Proceedings of the International Association of Hydrological Sciences (piahs)
% Primate Biology (pb)
% Scientific Drilling (sd)
% SOIL (soil)
% Solid Earth (se)
% The Cryosphere (tc)
% Web Ecology (we)
% Wind Energy Science (wes)


%% \usepackage commands included in the copernicus.cls:
%\usepackage[german, english]{babel}
% \usepackage{tabularx}
%\usepackage{cancel}
%\usepackage{multirow}
%\usepackage{supertabular}
%\usepackage{algorithmic}
%\usepackage{algorithm}
%\usepackage{amsthm}
%\usepackage{float}
%\usepackage{subfig}
%\usepackage{rotating}


\begin{document}

\title{An apportionment method for the Oxydative Potential to the atmospheric PM
sources: application to a one-year study in Chamonix, France.}


% \Author[affil]{given_name}{surname}

\Author[1]{Weber}{Samu\"{e}l}
\Author[1]{Uzu}{Ga\"{e}lle}
\Author[1]{Calas}{Aude}
\Author[1,2]{Chevrier}{Florie}
\Author[2]{Besombes}{Jean-Luc}
\Author[1,3]{Charon}{Aur\'{e}lie}
\Author[1]{Salameh}{Dalia}
\Author[4]{Je\v{z}ek}{Irena}
\Author[4,5]{Mo\v{c}nik}{Gri\v{s}a}
\Author[1]{Jaffrezo}{Jean-Luc}

\affil[1]{Univ. Grenoble Alpes, CNRS, IRD, IGE (UMR 5001), F-38000 Grenoble, France.}
\affil[2]{Univ. Savoie Mont-Blanc, LCME, F-73 000 Chamb\'{e}ry, France.}
\affil[3]{IFFSTAAR, F-69675 Bron, France.}
\affil[4]{Aerosol d.o.o., Kamni\v{s}ka 41, 1000 Ljubljana, Slovenia.}
\affil[5]{Condensed Physics Department, Jo\v{z}ef Stefan Institute, Ljubljana, Slovenia.}

%% The [] brackets identify the author with the corresponding affiliation. 1, 2, 3, etc. should be inserted.



\runningtitle{Apportionment of OP sources in Chamonix, France}

\runningauthor{Weber et al.}

\correspondence{Ga\"{e}lle Uzu (gaelle.uzu@ird.fr)}



\received{}
\pubdiscuss{} %% only important for two-stage journals
\revised{}
\accepted{}
\published{}

%% These dates will be inserted by Copernicus Publications during the typesetting process.



\firstpage{1}

\maketitle



\begin{abstract}
The particulate matter (PM) induces cellular oxidative stress in vivo
and so lead to adverse health outcome. The oxidative potential (OP) of
the PM appears to be a more relevant proxy of the health impact of the
aerosol rather than the total mass concentration. However, the relative
contributions of the emission's sources of aerosols to the OP are still
poorly known. In order to better quantify the impact of each PM source
to the air quality, we sampled aerosols in a French city for one year
(year 2014, 115 samples). A coupled analysis with detailed chemical speciation
(more than 100 species, including organic and carbonaceous compounds,
ions, metals and aethalomether measurements) and two OP assays (ascorbic
acid (AA) and dithiothreitiol (DTT)) in a simulated lung fluid (SLF)
were performed in these samples. We developed in this study a new
statistical model using a coupled approach with Positive Matrix
Factorisation (PMF) and multiple linear regressions to attribute a
redox-activity per PM sources. Our results highlight the importance of
the Biomass burning and Vehicular sources to explain the observed OP for
both assays. In general, we see a different contribution of the sources when
considering the OP AA, OP DTT or the mass of the PM\textsubscript{10}.
Moreover, some significant differences are observed between the DTT and
AA tests that emphasized the chemical specificities of the two tests and
the need of a standardized approach for the future studies on
epidemiology or toxicology of the PM.
\end{abstract}

\end{document}
